\documentclass[12pt]{article}

\usepackage{enumerate}
\usepackage{rotating}
\usepackage{multicol}
\usepackage{multirow}
\usepackage{graphicx}
\usepackage{fullpage}
\usepackage{subfigure}
\usepackage{setspace}
\usepackage{listings}
\usepackage{lastpage}

\graphicspath{{./images/}}

% for references
\usepackage[pagebackref=false,colorlinks,linkcolor=blue,citecolor=magenta]{hyperref}
\usepackage[nottoc]{tocbibind}
\usepackage{fancyhdr}
\setlength{\headsep}{15pt}

\pagestyle{fancy}
\fancyhf{}
\lhead{\lr{Digital Image Processing}}
\rhead{تمرین صفرم}
\cfoot{صفحه \thepage\ از \pageref{LastPage}}
\lfoot{نیمسال مهر 00-99}
\rfoot{حمیدرضا ابوئی مهریزی}


% xepersian
\usepackage[extrafootnotefeatures]{xepersian}
\settextfont[Scale=1.4]{B Nazanin}
\setlatintextfont{Times New Roman}

\renewcommand{\labelitemi}{$\bullet$}

\begin{document}
	\doublespacing
	\begin{titlepage}
		\paragraph*{}
		\centering
			
			
			{\small به نام او}\\
			\vspace{1cm}
			\includegraphics[width=0.12\paperwidth]{aut.png}
			\hspace{1cm}
			\includegraphics[width=0.15\paperwidth]{DIP}
			\hspace{1cm}
			\includegraphics[width=0.12\paperwidth]{bme}\\
			\vspace{2cm}
			{\Huge پردازش تصویر}\\
			\vspace{2cm}
			{\large استاد : دکتر حامد آذرنوش}\\
			\vspace{0.5cm}
			{\small  دانشجو :‌ حمیدرضا ابوئی}\\
			\vspace{0.5cm}
			{\small شماره دانشجویی : 9733002}
			\vfill
			{\tiny نیمسال مهر 00-99}
	\end{titlepage}
	\thispagestyle{plain}
	\tableofcontents
	\newpage
	%\onehalfspacing
	\doublespacing
	\section{سوال اول}
		\paragraph{توضیحات تکمیلی روند کد}
			ابتدا یک متغیر برای جمع نهایی انتگرال تعریف میکنیم و مقدار آن را برابر $0$ قرار می دهیم سپس با استفاده از روش مستطيلي مقدار ها را با هم جمع ميكنيم و در نهايت بايد عدد نهايي را در طول گام ضرب كنيم تا جواب نهايي به آن بستگي نداشته باشد.
			در حقيقت ما در محاسبه ي انتگرال از فرمول:
			$\Sigma_{-3}^{4} sin(x^2) \Delta x $
			استفاده مي‌کنیم که باید در 
			$\Delta x$ 
			در انتها ضرب کنیم .
			
		\paragraph{ورودی برنامه}
		این برنامه ورودی ندارد
		\paragraph{خروجی برنامه}
	خروجی این برنامه 
	$\int_{-3}^{4} sin(x^2) dx$
	 با روش 
	 $\Delta x = \frac{1}{200}$
است.		
که جواب آن
 \lr{ 1.5224190753314377}
  به دست می‌آید.
  \newpage
  \section{سوال دوم}
  \paragraph{توضیحات تکمیلی روند کد}
  اولا یک تابع با ورودی آن عدد می‌نویسیم و برای چک کردن این که این عدد اول است یا خیر، آن بخش پذیری آن عدد را به اعداد 2 تا قبل از رادیکال عدد ورودی چک میکنیم. اگر اول بود، همانجا یک متغیر را علامت گذاری کند و نیازی به چک کردن اعداد دیگر نیست (برای افزایش سرعت) . و برای نشان دادن صحت آن، ما اعداد 2 تا 50 را در یک حلقه بررسی کردیم 
  \paragraph{ورودی برنامه}
  ورودی تابع، همان عددی است که میخواهیم اول بودن یا نبودن آن را بررسی کنیم. 
  \paragraph{خروجی برنامه}
خروجی برنامه متغیر بولین 
\lr{ True , Fulse }
است. که می تواند مانند خروجی استفاده شده در برنامه ،یک جمله ی اخباری باشد :
	\begin{flushleft}
		\lr{ 5 is a prime number: True}
	\end{flushleft}
 	\newpage
	\section{سوال سوم}
	\paragraph{توضیحات تکمیلی روند کد}
	
	همان‌گونه که می‌دانید، برای تغییر یک عدد در بازه ی 
	$(min,max)$
	به معادل همان عدد در بازه ی 
	$(MIN,MAX)$
	باید از فرمول 
	$ out = \frac{(in - min)(MAX-MIN)}{max-min}+MIN$
	استفاده کرد.
	
	بنابراین در تابع ، ابتدا به سراغ تک تک داده ها رفته و آن ها را	با این فرمول به بازه ی دلخواه منتقل می‌کنیم، در قسمت های بعدی، می‌توانیم برای یافتن کمینه و بیشینه آرایه، از 
	\lr{numpy.amin , numpy.amax}
	استفاده کنیم. 
	برای تغییر 
	\lr{dtype}
	یک آرایه با ابعاد آرایه ورودی می‌سازیم و 
	 \lr{dtype}
	 آن را 
	 \lr{uint8}
	 قرار می‌دهیم. 
	 
	 در قسمت سوم هم برای ساخت ورودی از تابع 
	 \lr{numpy.random.rand}
	 استفاده میکنیم  و آن را در 5.12 ضرب و در انتها 2.3- می‌کنیم تا بازه ی مورد نیاز ساخته شود .
	 سپس آن را به بازه ی 0 تا 255 برده و آن را با 
	 \lr{matplotlib}
	  و یا 
	  \lr{open cv}
	  نمایش می‌دهیم.
	   
	\paragraph{ورودی برنامه}
	آرایه سه بعدی

	\paragraph{خروجی برنامه}
	در قسمت های 1 و 2 ، آرایه ی بین 0 تا 255 و در قسمت سوم هم خروجی به صورت زیر خواهد بود:\\
	
	\centering
	\includegraphics[scale=2]{3-3.png}

	\newpage
	\raggedleft
	\section{سوال چهارم}
	
	\paragraph{توضیحات تکمیلی روند کد}
	در این قطعه کد، ابتدا ضریب جملات به صورت یک لیست به ما داده می‌شوند. در یک تابع، ما این لیست را گرفته و با استفاده از کتابخانه ی
	\lr{sympy}
عبارت جبری مرتبط با آن را می‌سازیم. 
در تابعی دیگر، ما لیست ورودی که ضرایب یک چندجمله ای است را میگیریم و مشتق آن را به صورت دستی حساب می کنیم .
سپس ما با استفاده از 
\lr{lambdify }
خروجی را به ازای ورودی حساب می‌کنیم و با استفاده از کتابخانه ی 
\lr{matplotlib}
آن را نمایش می‌دهیم و به آن خصوصیت های خاص می‌دهیم و تیتر و لیبل میزنیم .

	\paragraph{ورودی برنامه}
	ضرایب جملات چند جمله ای ورودی
	\paragraph{خروجی برنامه}
	یک نمودار حاوی شکل چندجمله ای و مشتق آن. برای مثال:
	\includegraphics[scale=1]{4.png}
\end{document}
