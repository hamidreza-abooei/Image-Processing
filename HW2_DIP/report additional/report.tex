\documentclass[12pt]{article}

\usepackage{enumerate}
\usepackage{rotating}
\usepackage{multicol}
\usepackage{multirow}
\usepackage{graphicx}
\usepackage{fullpage}
\usepackage{subfigure}
\usepackage{setspace}
\usepackage{listings}
\usepackage{lastpage}

\graphicspath{{./images/}}

% for references
\usepackage[pagebackref=false,colorlinks,linkcolor=blue,citecolor=magenta]{hyperref}
\usepackage[nottoc]{tocbibind}
\usepackage{fancyhdr}
\setlength{\headsep}{25pt}

\pagestyle{fancy}
\fancyhf{}
\lhead{\lr{Digital Image Processing}}
\rhead{تمرین امتیازی تمزین دوم}
\cfoot{صفحه \thepage\ از \pageref{LastPage}}
\lfoot{نیمسال مهر 00-99}
\rfoot{حمیدرضا ابوئی مهریزی}


% xepersian
\usepackage[extrafootnotefeatures]{xepersian}
\settextfont[Scale=1.4]{B Nazanin}
\setlatintextfont{Times New Roman}

\renewcommand{\labelitemi}{$\bullet$}

\begin{document}
	\doublespacing
	\begin{titlepage}
		\paragraph*{}
		\centering
			
			
			{\small به نام او}\\
			\vspace{1cm}
			\includegraphics[width=0.12\paperwidth]{aut.png}
			\hspace{1cm}
			\includegraphics[width=0.15\paperwidth]{DIP}
			\hspace{1cm}
			\includegraphics[width=0.12\paperwidth]{bme}\\
			\vspace{2cm}
			{\Huge پردازش تصویر}\\
			\vspace{2cm}
			{\large استاد : دکتر حامد آذرنوش}\\
			\vspace{0.5cm}
			{\small  دانشجو :‌ حمیدرضا ابوئی}\\
			\vspace{0.5cm}
			{\small شماره دانشجویی : 9733002}\\
			\vspace{0.5cm}
			{\small تمرین دوم}\\
			\vfill
			{\tiny نیمسال مهر 00-99}
	\end{titlepage}
	\thispagestyle{plain}
	%\tableofcontents
	\newpage
	%\onehalfspacing
	\doublespacing
	\section*{سوال امتیازی تمرین دوم}
		\paragraph{توضیحات تکمیلی روند کد}
		روندی که بنده برای به دست آوردن یک کد به دست آوردم به این صورت بود که ابتدا با سرچ کردن انواع ترکیب های « بهبود معیار کنتراست » ، به چندین مقاله و سایت دست یافتم ؛اما هیچکدام ‍پاسخ مورد نیاز را در پی نداشت. بنابراین با تفکر و همچنین بررسی انواع شاخص های آماری و امتحان آن در تصاویر داده شده به نظر میرسد این فرمول که به آن متوسط قدر مطلق انحرافات گفته می‌شود معیار مناسبی برای بررسی میزان کنتراست تصویرمان است.
		
		\begin{equation}
		\label{eq1}
			absDerivation =\frac{ \sum\limits_{k=0}^{L-1} |z_k-mean|}{m\times n}
		\end{equation} 
		\begin{itemize}
			\item
		مزایا:\\
		استفاده از یک نقطه‌ مرکزی برای سنجش انحرافات\\
		نقش همه مقدارها در محاسبه پراکندگی\\
		\item
		معایب:\\
		تاثیرپذیری از مقدارهای خیلی بزرگ و یا خیلی کوچک\\
		وابستگی به واحد اندازه‌گیری داده‌ها\\
		پیچیدگی در استفاده از روش‌های ریاضی\\
		
		\end{itemize}
		\paragraph{ورودی برنامه}
		دو تصویر زیر ورودی برنامه است \\
		\vspace{0.5cm}\\
		\includegraphics[width=17cm]{images/inputs/in.png}
		\paragraph{خروجی برنامه}
		دو عدد به عنوان متوسط قدر مطلق انحرافات\\
		\noindent
	\lr{1.7142857142857142}\\
	\lr{1.5419501133786853}
	
	\raggedleft
	
\end{document}
